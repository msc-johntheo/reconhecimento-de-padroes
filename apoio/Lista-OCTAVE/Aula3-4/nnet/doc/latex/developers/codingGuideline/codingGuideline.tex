\chapter{Coding Guideline}
Some genereal descriptions why a variable has a chosen name. This is valid for the complete
toolbox... or so :-)\\
Here is only the description of variable names, which aren't visible to the user. Visible names are
described in the User's Guide!\\
The variable identifiers are taken from \cite{4}. One difference is purposeful added. If a variable has only one letter, a second small letter will be added to make it searchable. Who has ever tried to search a variable called "t"?

\section{Variable identifier}

\begin{tabbing}
\hspace*{1em} \= \textcolor{blue}{Identifier} \hspace*{3em}\= \textcolor{blue}{Description:} \\ 
  \textbf{Aa} \> \> hold the network values after transfer function.\\
  blf \>  \> \textbf{b}atch \textbf{l}earning \textbf{f}unction \\
  btf  \>  \> \textbf{b}atch \textbf{t}rainings \textbf{f}unction \\
  \textbf{Jj} \>  \> Jacobi matrix \\
  \textbf{Nn}  \> \> hold the network values before transfer function.\\  						
  net	\> \> structure which holds the neural network informations \\
  pf \>  \> \textbf{p}erformance \textbf{f}unction \\
  \textbf{Pp}				\>						\>input matrix; nInputs x nDataSets  \\
  \textbf{Pr}	\>		\> input range, this is a Nx2 matrix, that's why the capitalized P \\
  trf \>  \> \textbf{tr}ansfer \textbf{f}unction \\
  \textbf{Tt} \>  \> target matrix, nTargets x nDataSets\\
  \textbf{ss}	\> \> row vector with numbers of neurons in the layers, for each layer, one entry \\
  \textbf{vE}	\> \> row vector with errors... size depends on network structure. \\
  VV \>  \> Validation structure \\
  \textbf{xx} \>  \> Weight vector in column format\\
  
\end{tabbing}

\subsection{Nn}
\textbf{Nn} is a cell array and has one entry for each layer. In reality, this will have 2 or 3 layers.\\
In \textbf{Nn\{1,1\}} are the values for the first hidden layer. The size of this matrix depends
on the number of neurons used for this layer.\\
In \textbf{Nn\{2,1\}} are the values for the second hidden layer or the output layer. The size of this matrix depends
on the number of neurons used for this layer and so on ...\\
\textbf{Nn\{x,1\}} where \textbf{x} can be $\infty$.\\

\subsection{Aa}
\textbf{Aa} is a cell array and has one entry for each layer.\\
In \textbf{Aa\{1,1\}} are the values for the first hidden layer. The size of this matrix depends
on the number of neurons used for this layer.\\
In \textbf{Aa\{2,1\}} are the values for the second hidden layer or the output layer. The size of this matrix depends
on the number of neurons used for this layer.\\
See \textbf{Nn} for a more detailed description.\\

\subsection{vE}
\textbf{vE} is also a cell array which holds in the last (second) element the error vector. It's not completly clear, why in the last (second) element.\\
The number of rows depends on the number of output neurons. For one output neuron, \textbf{vE} holds only one row, for 2 output neurons, this holds of course 2 rows, and so on. 

\subsection{Jj}
This is the short term for the Jacobi matrix.